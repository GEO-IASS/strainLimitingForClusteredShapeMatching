%%% template.tex
%%%
%%% This LaTeX source document can be used as the basis for your technical
%%% paper or abstract.

%%% The parameter to the ``documentclass'' command is very important.
%%% - use ``review'' for content submitted for review.
%%% - use ``preprint'' for accepted content you are making available.
%%% - use ``tog'' for technical papers accepted to the TOG journal and
%%%   for presentation at the SIGGRAPH or SIGGRAPH Asia conference.
%%% - use ``conference'' for final content accepted to a sponsored event
%%%   (hint: If you don't know, you should use ``conference.'')

\documentclass[review]{acmsiggraph}
\usepackage{amsmath}
\usepackage{amssymb}
\usepackage{wasysym}
\usepackage[scaled=.92]{helvet}
\usepackage{times}
\usepackage{graphicx}
\usepackage{parskip}
\usepackage{url}
\usepackage[labelfont=bf,textfont=it]{caption}
\usepackage{color}
\usepackage{algorithm}
\usepackage{algorithmic}
\usepackage{enumitem}
\usepackage{authblk}

%----------------------------------------------------------------------------
%----------------------------------------------------------------------------
\definecolor{AdamColor}{rgb}{0,0,0.7}
\definecolor{BenColor}{rgb}{0,0.7,0}
\definecolor{tamarColor}{rgb}{0.8,0,0.8}
\definecolor{JoshColor}{rgb}{0.0,0.7,0.8}
\newcommand{\mycomment}[1]{}
\newcommand{\adam}[1]{{\color{AdamColor} #1}}
\newcommand{\ben}[1]{{\color{BenColor} #1}}
\definecolor{nilsCol}{rgb}{0.85,0.35,0.25}
\newcommand{\Nils}[1]{\textcolor{nilsCol}{#1}}
\newcommand{\tamar}[1]{\textcolor{tamarColor}{#1}}
\newcommand{\josh}[1]{\textcolor{JoshColor}{#1}}
% for final
%\newcommand{\adam}[1]{{#1}}
%\newcommand{\Nils}[1]{{#1}}

%\let\shortcite=\cite
%\newcommand{\shortcite}[1]{\cite{#1}}
\newcommand{\etal}{and colleagues}
\newcommand{\Mueller}{M\"uller~}
\newcommand{\BM}[1]{\B{#1}}
%\newcommand{\B}[1]{\mbox{\boldmath$#1$}}
%\newcommand{\B}[1]{\textbf{\textit{#1}}}
\newcommand{\B}[1]{\mathit{\mathbf{#1}}}
\newcommand{\Per}{\%}
\newcommand{\Unit}[1]{{\mbox{$\,\mathrm{#1}$}}}
\newcommand{\Snit}[1]{{\mbox{\small$\mathrm{#1}$}}}
\newcommand{\Tr}[1]{\mathrm{Tr}\left(#1\right)}
\newcommand{\sign}[1]{\mathrm{sign}\left(#1\right)}
\newcommand{\Hz}{\Unit{Hz}}
\newcommand{\MHz}{\Unit{MHz}}
\newcommand{\GHz}{\Unit{GHz}}
\newcommand{\Sec}{\Unit{sec}}
\newcommand{\SPF}{\Unit{sec/frame}}
\newcommand{\Min}{\Unit{min}}
\newcommand{\Max}{\Unit{max}}
\newcommand{\M}{\Unit{m}}
\newcommand{\Nab}{\B{\nabla}}
\newcommand{\TP}{^\mathsf{T}}

\newcommand{\Dist}{\mbox{dist}}

\newcommand{\figureTopBot}[1]{
  \begin{figure}[!tb]{\sloppy #1}\end{figure}
}

\newcommand{\figureTop}[1]{
  \begin{figure}[!t]{\sloppy #1}\end{figure}
}
 
\newcommand{\figureBot}[1]{
  \begin{figure}[!b]{\sloppy #1}\end{figure}
}

\newcommand{\figureWideTop}[1]{
  \begin{figure*}[!t]{\sloppy #1}\end{figure*}
}

\newcommand{\figureWideBot}[1]{
  \begin{figure*}[!b]{\sloppy #1}\end{figure*}
}

\newcommand{\eqAlgn}{\!\!&\!\!}

\newcommand{\Eref}[1]{Equation~(\ref{#1})}
\newcommand{\Erefs}[2]{Equations~(\ref{#1}) and (\ref{#2})}
\newcommand{\eref}[1]{Equation~(\ref{#1})}
\newcommand{\erefs}[2]{Equations~(\ref{#1}) and (\ref{#2})}
\newcommand{\Sref}[1]{Section~\ref{#1}}
\newcommand{\sref}[1]{Section~\ref{#1}}
\newcommand{\srefs}[2]{Sections~\ref{#1} and~\ref{#2}}
\newcommand{\fref}[1]{Figure~\ref{#1}}
\newcommand{\frefAND}[2]{Figures~\ref{#1} and~\ref{#2}}
\newcommand{\frefs}[2]{Figures~\ref{#1} and~\ref{#2}}
\newcommand{\frefss}[3]{Figures~\ref{#1}, \ref{#2}, and~\ref{#3}}
\newcommand{\frefsss}[4]{Figures~\ref{#1}, \ref{#2}, \ref{#3}, and~\ref{#4}}
\newcommand{\Fref}[1]{Figure~\ref{#1}}
\newcommand{\Frefs}[2]{Figures~\ref{#1} and~\ref{#2}}
\newcommand{\Frefss}[3]{Figures~\ref{#1}, \ref{#2}, and~\ref{#3}}
\newcommand{\Frefsss}[4]{Figures~\ref{#1}, \ref{#2}, \ref{#3}, and~\ref{#4}}
\newcommand{\tref}[1]{Table~\ref{#1}}

\renewcommand{\labelenumi}{\arabic{enumi}.}
\renewcommand{\labelenumii}{\alph{enumii}.}
\renewcommand{\labelenumiii}{\roman{enumiii}.}

\newenvironment{algstep}{%
  \begin{enumerate}%
    \setlength{\itemsep}{0in}%
    \setlength{\partopsep}{0in}%
    \setlength{\topsep}{0in}%
}{\end{enumerate}}

%----------------------------------------------------------------------------
%----------------------------------------------------------------------------

%%% Make the ``BibTeX'' word pretty...

\def\BibTeX{{\rm B\kern-.05em{\sc i\kern-.025em b}\kern-.08em
    T\kern-.1667em\lower.7ex\hbox{E}\kern-.125emX}}

%%% Used by the ``review'' variation; the online ID will be printed on 
%%% every page of the content.

\TOGonlineid{}

%%% Used by the ``preprint'' variation.

\TOGvolume{0}
\TOGnumber{0}

\title{Ductile Fracture for Clustered Shape Matching}
\author[*]{Ben Jones}
\author[**]{Joshua A. Levine}
\author[***]{Tamar Shinar}
\author[*]{Adam W. Bargteil}
\affil[*]{University of Utah}
\affil[**]{Clemson University}
\affil[***]{University of California, Riverside}
\pdfauthor{}

\keywords{Shape Matching, Ductile Fracture}

\begin{document}

%%% This is the ``teaser'' command, which puts an figure, centered, below 
%%% the title and author information, and above the body of the content.

 \teaser{
   %\includegraphics[height=1.5in]{Figures/}
   \caption{}
 }

\maketitle

\begin{abstract}

\end{abstract}

\begin{CRcatlist}
  \CRcat{I.3.7}{Computer Graphics}{Three-Dimensional Graphics and Realism}{Animation};
  \CRcat{I.6.8}{Simulation and Modeling}{Types of Simulation}{Animation}.
\end{CRcatlist}

\keywordlist

%% Required for all content. 

\copyrightspace

\section{Introduction}\label{sec:Introduction}

A decade ago, \Mueller and colleagues~\shortcite{Mueller:2005:MDB} introduced
{\em shape matching} for physics-based computer animation.  In this approach
objects are discretized into a set of particles, $p_i\in\mathcal{P}$, with rest positions, $\B{r}_i$, 
that follow a path, $\B{x}_i(t)$, in world-space through time.  
At each frame, shape matching solves for the rotation matrix, $\B{R}$, and a translation
vector, $\B{x}_{cm}-\B{r}_{cm}$, that minimize
\begin{equation}
\sum_i \left(\B{R}\left(\B{r}_i - \B{r}_{cm}\right)-\left(\B{x}_i-\B{x}_{cm}\right)\right)^2.
\end{equation}
The best translations are given by the center-of-mass in the rest and world space, respectively.  
The rotation, $\B{R}$, is computed through a polar decomposition.  Intuitively, this computation
yields the least-squares best-fit rigid transformation from the rest pose to the current deformed pose.
This transformation allows us to define goal positions, $\B{g}_i$,
\begin{equation}
\B{g}_i = \B{R}\left(\B{r}_i-\B{r}_{cm}\right)+\B{x}_{cm}.
\end{equation}
Hookean springs are then used to define forces that move the particles toward the goal positions.
This basic approach was extended by including linear and quadratic global deformations, cluster-based
deformations, and plasticity.

In their seminal work, \Mueller and colleagues~\shortcite{Mueller:2005:MDB} introduced a simple plasticity
model, but did not address fracture.  In follow up work, Rivers and James~\shortcite{Rivers:2007:FFL}
incorporated a simple fracture model, but did not address plasticity.  

In this paper, we present plasticity and fracture models inspired by finite element approaches to 
animating deformable bodies.  
Specifically, inspired by the work of Irving and colleagues~\shortcite{Irving:2004:IFE} and Bargteil and 
colleagues~\shortcite{Bargteil:2007:AFE}, we introduce a multiplicative plasticity model that incorporates
yield stress, flow rate, and work hardening.  Inspired by the work of O'Brien and colleagues~\shortcite{Obrien:1999:GMA,Obrien:2002:GMA},
we introduce a cluster-based fracture approach that splits individual clusters along the plane orthogonal to the direction the 
cluster is most stretched.
Taken together these contributions allow animation of ductile in the clustered shape matching framework.

\section{Related Work}
The geometrically motivated shape matching approach was introduced by \Mueller and 
colleagues~\shortcite{Mueller:2005:MDB}, who demonstrated impressive results and 
described the key advantages of the approach: efficiency, stability, and controllability.
Given these advantages, shape matching is especially appealing in interactive animation contexts such 
video games.  The authors also introduced several extensions including linear and quadratic deformations 
(in addition to rigid deformations), cluster-based deformation, and plasticity.  

Two years later, Rivers and James~\shortcite{Rivers:2007:FFL} introduced lattice-based shape matching,
which used a set of hierarchical lattices to define the shape matching clusters.  They took advantage
of the regular structure of the lattices to achieve extremely high performance.  They also incorporated a 
simple fracture model that removed over-extended links in the lattice.  

Since this time, the shape matching framework has been largely disregarded by the research community in favor of position-based
dynamics~\cite{Mueller:2007:PBD,Bender:2013:PBM,Bender:2014:ASO,Macklin:2014:UPP}.  One notable exception is
the work of Bargteil and Jones~\shortcite{Bargteil:2014:SLF}, which incorporated strain-limiting into clustered shape matching.
We incorporate our ductile fracture approach into their framework.

Ductile fracture is distinguished from brittle fracture by the inclusion plastic deformation.  Materials undergoing ductile
fracture (e.g. play dough) appear to {\em tear}, while brittle materials (e.g. glass) appear to {\em shatter}.  For many
practical fracture applications some amount of plastic deformation is needed to achieve realism.  Plasticity and fracture
were first demonstrating in computer animation by Terzopolous and colleagues~\shortcite{}, though they did not combine
these phenomena.  The first demonstration of ductile fracture was performed by O'Brien and colleagues~\shortcite{Obrien:2002:GMA},
who built on the finite element approach to brittle fracture of O'Brien and Hodgins~\shortcite{Obrien:1999:GMA}.

\section{Methods}

\section{Results and Discussion}

\paragraph{Limitations}

\section*{Acknowledgements}
Removed for anonymous review.

\bibliographystyle{acmsiggraph}
\bibliography{csm}
\end{document}
